\chapter{FAQ}

\section{Why do I need a test file during training?}

When you extract a grammar using Joshua's suffix array grammar extraction code, you are extracting a grammar that is specific to a particular test set. Think of this as extracting a grammar, then filtering so that you get only those rules which will actually be used to translate a particular test set.

No filtering actually takes place. More precisely, only those rules which will be used to translate the test set are extracted.

This means that you will need to extract a new grammar for each test set that you translate.


\section{Can I do online rule extraction instead?}

Right now, the only way to get a grammar is to extract a grammar for a test set before you translate (using joshua.prefix\_tree.ExtractRules), then list that grammar in the config file that you use when translating with joshua.decoder.JoshuaDecoder.

However, the suffix array rule extraction code is capable of doing online rule extraction. However, that code has not yet been fully integrated into the JoshuaDecoder. That's something that we are actively working on, and that feature will definitely be implemented.

Once that feature is implemented, instead of specifying a grammar when translating, you will be able to translate by specifying the source, target, and alignment files of your parallel training corpus, and the test set to be translated. The decoder will then use the suffix array code to extract rules on the fly for the sentences that are being translated.

